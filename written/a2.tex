\documentclass[fontsize=11pt]{article}
\usepackage{amsmath}
\usepackage[utf8]{inputenc}
\usepackage[margin=0.75in]{geometry}

\title{CSC110 Fall 2021 Assignment 2: Logic, Constraints, and Nested Data}
\author{Hung, Shou-Yi (Ray), Scott Cui}
\date{\today}

\begin{document}
\maketitle

\section*{Part 1: Predicate Logic}

\begin{enumerate}

    \item[1.]
        \begin{enumerate}
            \item[1.] If $D_1$ is the set of all Natural Numbers, then it's possible to only satisfy one statement. In this case, statement 1 will be satisfied, but statement 2 will not. Because for every number you pick in the Natural Numbers set, you can always find a number that is greater than it just by adding 1. But for the number 1 (or 0 depending on how you start the Natural Numbers), you cannot find a number that is smaller than it (Statement 2).
            \item[2.] If $D_2$ is the set of all Integers, including negatives, then it's possible to satisfy both. In this case, statement 1 will be satisfied because you can always find a number (y) greater than any number (x) you pick from the set. You can also always find a number (x) that is smaller than any number (y) you pick from the set.
            \item[3.] If $D_3$ is a set with only 1 element, then it's possible to falsify both statements. Pick $D_3 = {1}$, then for statement 1 and 2, we cannot find a y or x that is greater than or lesser than it. So both statements automatically becomes false.
        \end{enumerate}

    \item[2.]
        \begin{enumerate}
            \item[1.] $P(x): x \ge 6$
            \item[2.] $Q(x): x \ge 5$
        \end{enumerate}
            For all x in S, Statement 3 is False because when $x < 5$, $Q(x)$ is False.

            For all x in S, Statement 4 is True because the $P(x)$ covers $Q(x)$
            (when $P(x)$ is True, $Q(x)$ is always True.)
            , so whenever $P(x)$ is True or False, the entire conditional statement is True.

            Also, $P(x)$ and $Q(x)$ is not independent from $S$.
            
            Therefore, we found a $P(x)$ and $Q(x)$ that will make one of the statements False
            and the other statement True.
    \item[3.]
        Complete this part in the provided \texttt{a2\_part1.py} starter file.
        Do \textbf{not} include your solution in this file.

    \item[4.]
        Complete this part in the provided \texttt{a2\_part1.py} starter file.
        Do \textbf{not} include your solution in this file.

\end{enumerate}

\section*{Part 2: Conditional Execution}

Complete this part in the provided \texttt{a2\_part2.py} starter file.
Do \textbf{not} include your solution in this file.

\newpage

\section*{Part 3: Generating a Timetable}

\begin{enumerate}

    \item[1.]
        Complete this part in the provided \texttt{a2\_part3.py} starter file.
        Do \textbf{not} include your solution in this file.

    \item[2.]

        \begin{enumerate}
            \item[(a)]

                \emph{IMPORTANT DEFINITIONS/NOTATION} (don't change this text!)

                We define the following sets:

                \begin{itemize}
                    \item $C$: the set of all possible courses
                    \item $S$: the set of all possible sections
                    \item $M$: the set of all possible meeting times
                    \item $SC$: the set of all possible schedules
                \end{itemize}

                We also define the following notation for expressions involving the elements of these sets:

                \begin{itemize}
                    \item
                          The first three (courses/sections/meeting times) are represented as tuples (as described in the assignment handout), and you can use the indexing operation on these values. For example, you could translate ``every section term is in $\{'F', 'S', 'Y'\}$'' into predicate logic as the statement:

                          \[\forall s \in S,~ s[1] \in \{'F', 'S', 'Y' \} \]

                    \item
                          The start and end times of a meeting time can be compared chronologically using the standard $<$, $\leq$, $>$, and $\geq$ operators.

                    \item
                          For a section $s \in S$, $s[2]$ represents a tuple of meeting times.
                          You may use standard set operations and quantifiers for these tuples (pretend they are sets).
                          For example, we can say:

                          \begin{itemize}
                              \item $\forall s \in S,~ s[2] \subseteq M$
                              \item $\forall s \in S,~ \forall m \in s[2],~ m[1] < m[2]$
                          \end{itemize}

                    \item
                          Finally, for a schedule $sc \in SC$, you can use the notation $sc.sections$ to refer to a set of all sections in that schedule.
                          You can use quantifiers with that set of schedules as well, e.g.
                          $\forall s \in sc.sections,~ ...$
                \end{itemize}

                \textbf{Predicate for meeting times conflicting:}


                \begin{align*}
                    MeetingTimesConflict(m_1, m_2) : (m_1[0] = m_2[0]) \land                               &   \\
                    ((m_1[2] > m_2[1] \land m_2[2] > m_1[1]) \lor (m_2[2] > m_1[1] \land m_1[2] > m_2[1])) & , \\
                    \text{where $m_1, m_2 \in M$}                                                          &
                \end{align*}

                \smallskip

                \textbf{Predicate for sections conflicting:}


                \begin{align*}
                    SectionsConflict(s_1, s_2) : (s_1[1] = s_2[1] \lor (s_1[1] = 'Y' \lor s_2[1] = 'Y')) \land (\exists x \in s_1[2], \exists y \in s_2[2], & \\
                    MeetingTimesConflict(x, y)),                                                                                                                        & \\
                    \text{where $s_1, s_2 \in S$}                                                                                                                       &
                \end{align*}

                \smallskip

                \textbf{Predicate for valid schedule:}


                \begin{align*}
                    IsValidSchedule(sc) : \forall x \in sc.sections, \forall y \in sc.sections, x \neq y \Rightarrow \neg SectionsConflict(x, y), & \\
                    \text{where $sc \in SC$}                                                                                            &
                \end{align*}


            \item[(b)]
                Complete this part in the provided \texttt{a2\_part3.py} starter file.
                Do \textbf{not} include your solution in this file.
        \end{enumerate}

    \item[3.]

        \begin{enumerate}
            \item[(a)]

                You may use all notation from question 2(a).
                Note that a course $c \in C$ is a tuple, and $c[2]$ is a set of sections, and so can be quantified over: $\forall s \in c[2], ...$.

                \smallskip

                \textbf{Predicate for section-schedule compatibility:}


                \begin{align*}
                    IsCompatibleSection(sc, s) : \forall x \in sc.sections, \neg SectionsConflict(x, s) & \\
                    \text{where $sc \in SC, s \in S$}
                \end{align*}

                \smallskip

                \textbf{Predicate for course-schedule compatibility:}


                \begin{align*}
                    IsCompatibleCourse(sc, c) : \exists x \in c[2], IsCompatibleSection(sc, x) & \\
                    \text{where $sc \in SC, c \in C$}
                \end{align*}

            \item[(b)]
                Complete this part in the provided \texttt{a2\_part3.py} starter file.
                Do \textbf{not} include your solution in this file.
        \end{enumerate}

\end{enumerate}

\section*{Part 4: Processing Raw Data}
Complete this part in the provided \texttt{a2\_part4.py} starter file.
Do \textbf{not} include your solution in this file.

\end{document}
